\documentclass{beamer}
\usepackage{graphicx}
\usepackage{subfig}
\usepackage{hyperref}
\usepackage{fancyvrb}
\usepackage[T1]{fontenc} % recommended for languages with accents
\usepackage[utf8]{inputenc}
\usetheme{Darmstadt}

\title{Topic Detection and Tracking}
\author{
	Marc-André Faucher\\
	Jeff How\\
Jonathan Villemaire-Krajden}
\date{June 6, 2013}

\begin{document}

\begin{frame}[plain]
	\section{Introduction}
	\frametitle{Example}
	\titlepage
\end{frame}

\begin{frame}[plain]
	\frametitle{Terminology}
\end{frame}

\begin{frame}
	\frametitle{Subtasks}
	\begin{itemize}
		\item Topic Detection
			\begin{itemize}
				\item 1st story
			\end{itemize}
		\item Topic Tracking
			\begin{itemize}
				\item Finding additional stories about a particular topic
				\item Clustering
			\end{itemize}
	\end{itemize}
\end{frame}

\section{Implementation}
\begin{frame}
	\frametitle{References}
\end{frame}

\begin{frame}
	\frametitle{Feature Selection}
	\begin{itemize}
		\item Lexical: Words, Noun Phrases, Named Entities
		\item Syntactic: POS tagging: nouns, verbs, proper names)
		\item Semantic: Temporal language cues (verb tense and temporal NPs)\cite{Makkonen:2003:UTITDT}
		\item Metadata: Timestamps
	\end{itemize}
\end{frame}

\begin{frame}
	\frametitle{Model Selection}
\end{frame}

\begin{frame}
	\frametitle{Cluster}
	Stories are grouped through clustering using distance metrics based on the
	model (probabilistic, cosine similarity, etc.):
	\begin{columns}[c]
		\column{3in}
		\begin{itemize}
			\item {\bf K-means:}
				\begin{itemize}
					\item Selecting small \emph{k} with small variance
					\item Centroid represents dominant features
				\end{itemize}
			\item {\bf Hierarchical agglomerative clustering:}
				\begin{itemize}
					\item Provides a hierarchy of clusters 
				\end{itemize}
			\item {\bf KeyGraph link based clustering:}
				\begin{itemize}
					\item Network of features and relations
					\item Topics are identified using network theory
				\end{itemize}
			\item Advanced Techniques:
				\begin{itemize}
					\item Latent Dirichlet Allocation (LDA)
					\item Non-negative Matrix Factorization
				\end{itemize}
		\end{itemize}
		\column{1in}
			\includegraphics[width=\textwidth]{img/cluster_centroid.png} \\
			\hfill \\
			\includegraphics[width=\textwidth]{img/cluster_link.png} \\
	\end{columns}
\end{frame}

\subsection{Examples}
\begin{frame}
	\frametitle{Time-Span \cite{Swan:1999:EST:319950.319956}}
	% TODO: (note to self: this section needs work)
	\begin{itemize}
		\item Entities \& noun phrases with overlapping time-spans constitutes a topic.
		\item Calculate probabilities of time-span overlap for features if independence is assumed.
		\item Merge features, which are statistically dependent ($\chi^2$) in terms of doc occurrence.
	\end{itemize}
\end{frame}

\subsection{Examples}
\begin{frame}
	\frametitle{Topic Tracking in Tweet Streams \cite{Lin:2011:STA:2020408.2020476}}
	\begin{itemize}
		\item High arrival rate (up to 4000+ tweets per second)
		\item Foreground model: tracks recent topic counts
			\begin{itemize}
				\item History of \emph{h} events
				\item Smoothed with the background model
			\end{itemize}
		\item Background model: long-term estimates of term distributions
			\begin{itemize}
				\item Handles sparsity from limited history of foreground model
			\end{itemize}
		\item Evaluation based off hashtags
	\end{itemize}
\end{frame}

\section{Conclusion}
\begin{frame}
	\frametitle{Conclusion}
	% TODO: (note to self: this section needs work)
	\begin{itemize}
		\item Topic Detection
		\item Topic Tracking
		\item Implementation: Feature -> Model -> Cluster
		\item Current Application:
			\begin{itemize}
				\item EMM
				\item Google News
			\end{itemize}
	\end{itemize}
\end{frame}

\begin{frame}
	% TODO: Fix citation, alse reference style may be too large
	\frametitle{References}
	\bibliographystyle{plain}
	\bibliography{ref}
\end{frame}

\end{document}


%% \begin{frame}
%% \frametitle{EXAMPLE IMAGE}
%% Some stuff references the image~\ref{fig:example} which can be found bellow.
%% \begin{figure}[h]
%% \centering
%% \includegraphics[width=\textwidth]{opencalais}
%% \caption{An example.}
%% \label{fig:example}
%% \end{figure}
%% \end{frame}
